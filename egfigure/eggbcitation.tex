\documentclass{article}
\usepackage{ctex}
\usepackage{xcolor}
\usepackage{toolbox}
\usepackage[colorlinks]{hyperref}
\usepackage{lipsum}
\usepackage[paperwidth=21cm,paperheight=29cm,top=3cm,bottom=2cm,left=1.5cm,right=1.5cm]{geometry}
\usepackage{xltxtra,mflogo,texnames}
\usepackage[backend=biber,style=gb7714-2015]{biblatex}%sorting=nyt

%\usepackage{filecontents}
%\begin{filecontents}{\jobname.bib}
%
%\end{filecontents}

\addbibresource{example.bib}
\renewcommand{\thefootnote}{\textcircled{\tiny\arabic{footnote}}}


\begin{document}

\section*{GB/T 7714-2015 中的著录标准和顺序编码制示例}

\subsection*{4.1 专著}
\begin{refsection}

\nocite{陈登原2000-29-29,
哈里森沃尔德伦2012-235-236,
北京市政协民族和宗教委员会2012-112-112,
全国信息与文献标准化技术委员会2010-2-3,
徐光宪2010--,
顾炎武1992--,
王夫之1865--,
牛志明2012--,
中国第一历史档案馆2001--,
杨保军2012--,
赵学功2001--,
同济大学土木工程防灾国家重点实验室2011-5-6,
中国造纸学会2003--,
Peebles2001--,
Yufin2000--,
Baldock2011-105-105,
Fan2013-25-26
}

\printbibliography[heading=subbibliography]
\end{refsection}

\subsection*{4.2 专著中的析出文献}
\begin{refsection}

\nocite{
王夫之2011-1109-1109,
程根伟1999-32-36,
陈晋镳1980-56-114a,
马克思2013-302-302,
贾东琴2011-45-52,
Weinstein1974-745-772,
Roberson2011-1-36
}

\printbibliography[heading=subbibliography]
\end{refsection}

\subsection*{4.3 连续出版物}
\begin{refsection}

\nocite{
中华医学会湖北分会1984----,
中国图书馆学会1957--1990--,
AAAS1883----,
}

\printbibliography[heading=subbibliography]
\end{refsection}

\subsection*{4.4 连续出版物中的析出文献}
\begin{refsection}

\nocite{袁训来2012-3219-3219,
余建斌2013--,
李炳穆2008-6-12,
李幼平2010-225-228,
武丽丽2008-8-9,
Kanamori1998-2063-2063,
Caplan1993-61-66,
Frese2013-378-398,
Myburg2014-356-362
}

\printbibliography[heading=subbibliography]
\end{refsection}

\subsection*{4.5 专利文献}
\begin{refsection}

\nocite{邓一刚2006--,
西安电子科技大学2002--,
Tachibana2005--}

\printbibliography[heading=subbibliography]
\end{refsection}

\subsection*{4.6 电子资源}
\begin{refsection}

\nocite{中国互联网络信息中心2012--,
北京市人民政府办公厅2005--,
Bawden2008--,
OCLC--,
Hopkinson2009--
}

\printbibliography[heading=subbibliography]
\end{refsection}


\subsection*{8.6 获取和访问路径}
\begin{refsection}

\nocite{储大同2010-721-724,weiner2010-38}

\printbibliography[heading=subbibliography]
\end{refsection}

\subsection*{8.7 数字对象唯一识别符}
\begin{refsection}

\nocite{刘乃安2000-17-18,Deverell2013-21-22}

\printbibliography[heading=subbibliography]
\end{refsection}

\subsection*{9.2 文献表}
\begin{refsection}

\nocite{Baker1995--,Chernik1982--,尼葛洛庞帝1996--,汪冰1997-16-16,杨宗英1996-24-29,Dowler1995-5-26}

\printbibliography[heading=subbibliography]
\end{refsection}

\subsection*{10.1.1 一处引用一篇文献}

\begin{refsection}

所谓移情,就是“说话人将自己认同于......他用句子所描述的时间或状态中的一个参与者”\cite{Sunstein1996-903-903}。《汉语大词典》和张相
\cite{Morri2010--}都认为“可”是“痊愈”,
候精一认为是“减轻”\cite{罗杰斯2011-15-16}。......另外,根据候精一,表示病痛程度减轻的形容词“可”和表示逆转否定的副词“可”
是兼类词\cite{陈登原2000-29-29},这也说明二者应该存在着源流关系。


所谓移情,就是“说话人将自己认同于......他用句子所描述的时间或状态中的一个参与者”\footfullcite{Sunstein1996-903-903}。《汉语大词典》和张相
\footfullcite{Morri2010--}都认为“可”是“痊愈”,
候精一认为是“减轻”\footfullcite{罗杰斯2011-15-16}。......另外,根据候精一,表示病痛程度减轻的形容词“可”和表示逆转否定的副词“可”
是兼类词\footfullcite{陈登原2000-29-29},这也说明二者应该存在着源流关系。


\end{refsection}

\subsection*{10.1.2 一处引用多篇文献}
\begin{refsection}

裴伟提出\cite{Humphrey1971--,CRANE1972--}......

莫拉德对稳定区的研究
\cite{CRANE1972--,WEINSTEIN1974-745-772,KENNEDY1975-311-386}......


\end{refsection}

\subsection*{10.1.3 多次引用同一著者的同一文献}
\begin{refsection}
……改变社会规范也可能存在类似的“二阶囚徒困境”问题:尽管改变旧的规范对所有人都好,但个人理性选择使得没有人愿意率先违反旧的规范\cite{Sunstein1996-903-903}。
……事实上,古希腊对轴心时代思想真正的贡献不是来自对民主的赞扬,而是来自对民主制度的批评,苏格拉底、柏拉图和亚里士多德3位贤圣
都是民主制度的坚决反对者\pagescite[20]{Morri2010--}。
……柏拉图在西方世界的影响力是如此之大以至于有学者评论说,一切后世的思想都是一系列为柏拉图思想所作的脚注\cite{罗杰斯2011-15-16}。
……据《唐会要》记载,当时拆毁的寺院有4 600余所,招提、兰若等佛教建筑4万余所,没收寺产,并强迫僧尼还俗达260 500人。
佛教受到极大的打击\pagescite[326-329]{Morri2010--}。
……陈登原先生的考证是非常精确的,他印证了《春秋说题辞》“黍者绪也,故其立字,禾入米为黍,为酒以扶老,为酒以序尊卑,禾为柔物,亦宜养老”,指出:“以上谓等威之辨,尊卑之序,由于饮食荣辱。”\cite{陈登原2000-29-29}

\printbibliography[heading=subbibliography]
\end{refsection}

\begin{refsection}
……改变社会规范也可能存在类似的“二阶囚徒困境”问题:尽管改变旧的规范对所有人都好,但个人理性选择使得没有人愿意率先违反旧的规范
\footfullcite{Sunstein1996-903-903}。
……事实上,古希腊对轴心时代思想真正的贡献不是来自对民主的赞扬,而是来自对民主制度的批评,苏格拉底、柏拉图和亚里士多德3位贤圣
都是民主制度的坚决反对者\footfullcite[20]{Morri2010--}。
……柏拉图在西方世界的影响力是如此之大以至于有学者评论说,一切后世的思想都是一系列为柏拉图思想所作的脚注\footfullcite{罗杰斯2011-15-16}。
……据《唐会要》记载,当时拆毁的寺院有4 600余所,招提、兰若等佛教建筑4万余所,没收寺产,并强迫僧尼还俗达260 500人。
佛教受到极大的打击\footfullcite[326-329]{Morri2010--}。
……陈登原先生的考证是非常精确的,他印证了《春秋说题辞》“黍者绪也,故其立字,禾入米为黍,为酒以扶老,为酒以序尊卑,禾为柔物,亦宜养老”,指出:“以上谓等威之辨,尊卑之序,由于饮食荣辱。”\footfullcite{陈登原2000-29-29}
\end{refsection}

\end{document}
