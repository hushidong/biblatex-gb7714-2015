\documentclass[twoside]{article}
\usepackage{ctex}
\usepackage{xcolor}
\usepackage{hyperref}
\usepackage{lipsum}
\usepackage[top=2cm,bottom=1cm,left=3cm,right=3cm]{geometry}

\usepackage[backend=biber,style=gb7714-2015]{biblatex}

\usepackage{filecontents}
\begin{filecontents}{\jobname.bib}
@ARTICLE{徐伟康2010对,
  AUTHOR = {徐伟康},
  DATE = {2010},
  JOURNALTITLE = {经济研究},
  NUMBER = {5},
  PAGES = {139--148},
  TITLE = {对《 消费者价格指数与生产者价格指数: 谁带动谁?》 一文的质疑},
  USERA = {J},
}

@ARTICLE{徐伟康2011,
  AUTHOR = {徐伟康 and others},
  DATE = {2011},
  JOURNALTITLE = {经济研究},
  NUMBER = {5},
  PAGES = {139--148},
  TITLE = {对《 消费者价格指数与生产者价格指数: 谁带动谁?》 一文的质疑},
  USERA = {J},
}

@ARTICLE{杨光2015经济波动,
  AUTHOR = {杨光 and 孙浦阳 and 龚刚 and 徐伟康},
  DATE = {2015},
  JOURNALTITLE = {经济研究},
  NUMBER = {2},
  PAGES = {47--60},
  TITLE = {经济波动, 成本约束与资源配置},
  VOLUME = {50},
}

@INPROCEEDINGS{FOURNEY1971-17-38,
  AUTHOR = {FOURNEY, M E},
  LOCATION = {New York},
  PUBLISHER = {ASME},
  BOOKTITLE = {Symposium on Applications of Holography in Mechanics, August 23-25, 1971, University of Southern California, Los Angeles, California},
  DATE = {1971},
  PAGES = {17--38},
  TITLE = {Advances in holographic photoelasticity},
}

@BOOK{Yi2013--,
  AUTHOR = {Yi, S H and Zhao, Y X and He, L and Zhang, M L},
  LOCATION = {BeiJing},
  PUBLISHER = {National Defense Industry Press},
  DATE = {2013},
  TITLE = {Supersonic and hypersonic nozzle design},
}
\end{filecontents}
    \addbibresource{\jobname.bib}
    %

    \begin{document}
    \section{set title}
    created with biblatex v\versionofbiblatex, last revised at \today; Style Files (gb7714-2015*.*) have version number: \versionofgbtstyle.

\bigskip
    GB/T 7714-2015 标准 10.1.1 节,引用单篇文献:

    文献\cite{徐伟康2010对}提到。文献\cite{FOURNEY1971-17-38}提到。(use cite)\par
    文献\parencite{徐伟康2010对}提到。文献\parencite{FOURNEY1971-17-38}提到。(use parencite)\par
    \textcite{徐伟康2010对}提到。\textcite{FOURNEY1971-17-38}提到。(use textcite)\par
    \authornumcite{徐伟康2010对}提到。\authornumcite{FOURNEY1971-17-38}提到。
    (use authornumcite)\par
    \textcite{徐伟康2011}提到。(use textcite)

\bigskip
    GB/T 7714-2015 标准 10.1.2 节,引用多篇文献:

    文献\cite{杨光2015经济波动,Yi2013--}提到。(use cite)\par
    文献\parencite{杨光2015经济波动,Yi2013--}提到。(use parencite)\par
    \textcite{杨光2015经济波动,Yi2013--}提到。(use textcite)\par
    \authornumcite{杨光2015经济波动,Yi2013--}提到。(use authornumcite)\par


\bigskip
    GB/T 7714-2015 标准 10.1.3 节,多次引用文献:

    文献\pagescite[16]{杨光2015经济波动}提到。
    文献\pagescite[16]{徐伟康2010对}提到。
    文献\pagescite[16]{Yi2013--}提到。
    文献\pagescite[16]{FOURNEY1971-17-38}提到。(use pagescite)\par
    \authornumcite{杨光2015经济波动}提到。
    \authornumcite{徐伟康2010对}提到。
    \authornumcite{Yi2013--}提到。
    \authornumcite{FOURNEY1971-17-38}提到。(use authornumcite)\par

\bigskip
    其它用法:

    文献\cite[见][49页]{杨光2015经济波动}。
    文献\parencite[见][49页]{杨光2015经济波动}。
    见\citeauthor{杨光2015经济波动}\cite{杨光2015经济波动}。
    文献\footnote{脚注中引用文献\footcite{杨光2015经济波动}。}。
    文献\footfullcite{杨光2015经济波动}。


    \printbibliography

    \end{document} 