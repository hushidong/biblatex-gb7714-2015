\documentclass{article}
\usepackage{ctex}
\usepackage{xcolor}
\usepackage[margin=2cm]{geometry}
\usepackage[colorlinks]{hyperref}
\usepackage{xurl}
\usepackage[backend=biber,style=chinese-cajhssay]{biblatex}
\newcommand{\qd}[1]{\textbf{\textcolor{blue}{#1}}}

\begin{filecontents}[force]{\jobname.bib}

@inbook{霍弗特2005,
	title = {中世纪晚期的国家、城市与公民},
	author = {{阿尔穆特·霍弗特}},
	editor = {{昆廷·斯金纳} and {博·斯特拉思}},
	editortype = {主编},
	BookTitle= {国家与公民:历史·理论·展望},
    translator={彭利平},
	date = {2005},
	location = {上海},
	publisher = {华东师范大学出版社},
}

@inbook{蔡英文2006,
	title = {公民身份的多重性——政治观念史的阐述},
	author = {蔡英文},
	editor = {许纪霖},
	editortype = {主编},
	BookTitle= {公共性与公民观},
	date = {2006},
	location = {南京},
	publisher = {江苏人民出版社},
}

@book{韦伯1993,
	title = {儒教与**},
	address={南京},
	publisher = {江苏人民出版社},
	author = {{马克斯·韦伯}},
    shortauthor={{韦伯}},
    translator={洪天富},
	date = {1993},
}


@book{韦伯2004a,
	title = {韦伯作品集II:经济与历史·支配的类型},
	address={桂林},
	publisher = {广西师范大学出版社},
	author = {{马克斯·韦伯}},
    shortauthor={{韦伯}},
    translator={康乐 and others},
	date = {2004},
}

@book{韦伯2004b,
	title = {韦伯作品集III:支配社会学},
	address={桂林},
	publisher = {广西师范大学出版社},
	author = {{马克斯·韦伯}},
    shortauthor={{韦伯}},
    translator={康乐 and 简惠美},
	date = {2004},
}


@book{韦伯2005,
	title = {韦伯作品集:非正当性的支配——城市的类型学},
	address={桂林},
	publisher = {广西师范大学出版社},
	author = {{马克斯·韦伯}},
    shortauthor={{韦伯}},
    translator={康乐 and 简惠美},
	date = {2005},
}


@book{罗桑瓦龙2005,
	title = {公民的加冕礼——法国谱选史},
	address={上海},
	publisher = {上海世纪出版集团},
	author = {{皮埃尔·罗桑瓦龙}},
    translator={吕一民},
	date = {2005},
}

@article{唐兴霖1999,
	title = {中国村民自治民主的制度分析},
	issue = {第5、6月号},
	journaltitle = {开放时代},
	author = {唐兴霖 and 马骏},
	date = {1999},
}

@book{Huter1982,
	title = {Meta-analysis:Cumulating Research Findings Across Studies},
	address={Beverly Hills,CA},
	publisher = {Sage},
	author = {Huter, J. E. and F. L. Schmidt and G. B. Jackson},
	date = {1982},
}

@book{Isin2002,
	title = {Handbook of citizenship Studies},
	address={London},
	publisher = {SAGE Publications},
	editor = {Isin, Engin F. and  Bryan Turner},
	date = {2002},
}

@book{Marshall1964,
	title = {Class, Citizenship and Social Development},
	address={Chicago},
	publisher = {University of Chicago Press},
	author = {Marshall, T.H.},
	date = {1964},
}

@inbook{Turner2002,
	title = {Religion and Politics:The Elementary Forms of Citizenship},
	author = {Turner, Bryan S.},
	editor = {Engin Isin and Bryan Turner},
	BookTitle= {Handbook of Citizenship Studies},
	date = {2002},
	location = {London},
	publisher = {SAGE Publications},
}



\end{filecontents}

\addbibresource{\jobname.bib}


\begin{document}

\title{综合性期刊文献引证技术规范}
\maketitle

\section{著者—出版年体例}

“著者—出版年”体例是由正文中括注著者-出版年及页码的标识(作者,出版年代:页码),与文后“参考文献表”以及对正文中的术语、概念、观点和资料进行解
释、辨析或评论的注释三部分共同构成。注释与详列所引文献详细信息的参考文献表分开编的,这是该体例与注释体例在形式上的最大区别。

\section{引文的标识}

1.在引文处括注引文著者姓名(西文著作只需标引著者姓氏),出版年代和页码,一般情况下,圆括号标识放在所引材料的最后一个句子或短语的标点符号之前。
提行缩格的引文,在引语的最后一个标点符号后加圆括号注释。

示例1:
发展成熟的皇权官僚体制一直有效地行使着对城市的统辖职能,城市是代表皇权的官员所在的非自治地区,而村落倒是无官员的自治地区。“可以毫不夸张地说,中国
的治理史乃是皇权试图将其统辖势力不断扩展到城外地区的历史”(韦伯,1993:110)。

\cite[110]{韦伯1993}
\qd{注意:引用的标注标签在英文中通常使用姓,而中文则使用全名。一般的中英文姓名都能正常显示,但对于中间带点的英文姓名的中文译名(比如:马克斯·韦伯),默认也像中文姓名一般处理,所以会直接输出全名,但本样式要求仅输出“韦伯”。所以,需要给这类姓名提供一个shortauthor用于放置需要输出的标签内容,比如这里的“韦伯”,所以这类文献条目需要增加一个域:shortauthor=\{\{韦伯\}\}}

示例2:

在教权制里¼¼对于那些被它要求支配的人,教权制在其支配权所及的范围内,会保护他们免于来自其他权力的干涉,无论此一干涉为政治当权者、丈夫或**。此种力
量来自于教权制本身的官职卡里斯玛。(韦伯,2004b:438)

\cite[438]{韦伯2004b}\qd{注意:引用页码等信息默认作为postnote处理,放在条目的entrykey前面的可选框,比如:}\verb|\cite[438]{韦伯2004b}|


2.当作者姓名成为正文的组成部分时,可省去圆括号中的作者姓名。

示例:
按照英国社会学家马歇尔(Marshall)那引起广泛关注的观点,自18世纪以来,这种与公民资格相联系的公民权利大体上经历了从基本的“市民权利“(civil rights)到政治权利再到社会权利的发展历程(1964:56)。

\citeyear[56]{Marshall1964}
\qd{注意:对于不输出作者姓名,但输出年份和页码的文献,则使用citeyear命令。比如:}
\verb|\citeyear[56]{Marshall1964}|

3.两位作者姓名之间用顿号隔开。

示例1:
有的学者甚至明确认为:西方社会民主政治制度的基础是市民社会,而中国社会的特点决定了民主政治发展的基础是乡村社会(唐兴霖、马骏,1999)。

\cite{唐兴霖1999}

示例2:
从新古典共和主义之“公民”理想,到主权理论的“属民”概念,再到革命立宪的“人权”和“公民权”,乃至康德“世界公民”的观念,公民权概念的内涵一直处于
流变之中(蔡英文,2006)。即使到今天,自由主义、共和主义、社群主义以及激进民主理论等不同的理论取向对于这一概念的理解也依然各不相同(Isin \& Turner,
2002:131-188)。

\cite{蔡英文2006}\cite[131-188]{Isin2002}


4.引用同一个作者同一年的多篇论著时,在出版年代后加a,b,c等以示区别。

示例1:
但无论是教会还是教派,和所有具有**取向的**一样,都“在一种末世论期待的影响下,一开始即带有卡里斯玛式的拒斥现世的烙印”(韦伯,2004b:403)。

\cite[403]{韦伯2004b}

示例2:
在不同的场合,韦伯从城市的社会结构与阶级对立(贵族与农民、奴隶,抑或贵族与行会)、城市政治组织的社会基础(地区共同体,抑或职业团体)、早期民主制的
担纲者(农民,或是工商业市民)、经济政策的利益取向、身份结构等一系列方面对中世纪城市和古代城市进行了比较(韦伯,2005:第5章;2004a:233-245、272-278)。

(\citets[第5章]{韦伯2005}[233-245、272-278]{韦伯2004a})
%(\citet[第5章]{韦伯2005};\citet[233-245、272-278]{韦伯2004a})
%\cites[第5章]{韦伯2005}[233-245、272-278]{韦伯2004a}
%(\citet{韦伯2005,韦伯2004a})
\qd{注意:当多篇文献同时要处理页码等信息时,引用可以分开处理(使用多个cite,citet或parencite),也可以连着处理(使用cites,citets,parencites)。由于分开处理不会省略相同的作者,所以要省略相同的作者时,需要使用citets等。其中,cite和citet的差别是前者带有一个默认的外括号,所以使用cites时括号处理不一定适合,所以我们可以用citets不给出括号,然后自己给出括号。比如:}
\verb|(\citets[第5章]{韦伯2005}[233-245、272-278]{韦伯2004a})|

5.转引的文献须注明。

示例:
就像奥古斯丁·梯叶里在其著名的《论第三等级的历史》(1853)中所说的:“12世纪一系列的城市革命,从某种方面讲类似于为当今这么多国家带来立宪政府制度的
运动……资产阶级,一种以市民平等和劳动**为惯例的新国民,已经在贵族和人民之间升起,永远地摧毁了封建时代的社会两重性”(转引自霍弗特,2005:83)。

(转引自\citet[83]{霍弗特2005})。\qd{注意:类似有“转引自”这种附加信息的标注,因为不便于用命令直接给出,所以可以自己提供这些附加信息,然后利用citet提供不带括号的文献的引用标注信息,比如:}\verb|(转引自\citet[83]{霍弗特2005})|

\section{注释}
注释是对正文中的术语、概念、观点和资料进行解释、辨析或评论的文字,从功能上说,相当于注释体例中的“内容性注释”,形式也与之完全一致。

注释序号用①,②,③……,或用右上角码1,2,3……,通过注释序号将正文中需要注释之处与页**释准确对应联系。正文中的注释号通常置于包含术语、概念、观点的句子的标点符号之后。

如果注释**现直接或间接引文及需要参考的文献,标注方式与正文相同,即采行“著者—出版年“方式标注,同样,相应的参考文献详细信息应列入文后参考文献表中,排列顺序也相同。
注释可放在当页,也可放在正文后,参考文献表之前。


\section{参考文献的标注项目与顺序}

1.标注项目

每一条参考文献的标注项目与“注释体例”中“资料性注释”的标注项目大致相同。不同之处是参考文献的著录项目中没有页码一项(因为已经在正文中标识)。

2.标注顺序

与“注释体例”中“资料性注释”相比,参考文献标注项目中“出版年”置于责任者与题名之间外,并将“年”省去,均用逗号(英文参考文献用逗点)隔开,其他项
目的顺序与标注格式则与“资料性注释”各项标注相同。

3.参考文献表的排列顺序

所有征引的参考文献,按下列顺序排列于正文后:
(1)各条文献首先按文种集中,先中文文献,后外文文献。
(2)其次按著者姓氏字顺排列。中文著作署名按照作者姓氏汉语拼音母音序排列,外文文献中译本的署名也按照作者姓名的汉语拼音母音序排列,但是按照习惯,外
人姓名名在前,姓在后(尽管习惯上正文中仍然只标识姓氏);外文文献按照作者姓氏的字母顺序排列(姓氏置前,用“,“与名字隔开;多位作者只须将首位作者的姓氏前
置)。
(3)同一著者再按出版年先后排列。




\newpage
\nocite{*}
\printbibliography[heading=bibliography]
\end{document}


