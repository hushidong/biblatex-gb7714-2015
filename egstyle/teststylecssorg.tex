\documentclass{article}
    \usepackage{ctex}
    \usepackage{xcolor}
    \usepackage{toolbox}
    \usepackage[colorlinks]{hyperref}
    \usepackage{lipsum}
    \usepackage[top=3cm,bottom=3cm,left=3cm,right=3cm]{geometry}

\usepackage[backend=biber,style=chinese-css,gbfootbib=true,gbfnperpage=true,url=false]{biblatex} %gbfootbibfmt


\begin{filecontents}[force]{\jobname.bib}

@book{庄子天下,
title={庄子·天下},
}

@book{尚书泰誓,
title={尚书·泰誓},
}

@book{尚书皋陶谟,
title={尚书·皋陶谟},
}


@article{Chamberlain1993,
	title = {On the search for civil society in {China}},
	volume = {19},
	doi = {10/d5gx4j},
	number = {2},
	journal = {Modern China},
	author = {{Heath B. Chamberlain}},
	date = {1993-04},
	pages = {199--215},
}

@online{扬之水2007,
	title = {两宋茶诗与茶事},
	url = {http://www.literature.org.cn/Article.asp?ID=199},
	urldate = {2007-09-13},
	version = {《文学遗产通讯》(网络版试刊)2006年第1期},
	author = {扬之水},
}

@book{杨钟羲1991,
	address = {沈阳},
	edition = {影印本},
	title = {雪桥诗话续集},
	volume = {5},
	publisher = {辽沈书社},
	author = {杨钟羲},
	year = {1991},
    entrysubtype={classic},
	note = {Type: classic},
}

@book{姚际恒,
	edition = {光绪三年苏州文学山房活字本},
	title = {古今伪书考},
	volume = {3},
	author = {姚际恒},
    entrysubtype={classic},
	note = {Type: classic},
}

@book{晚清洋务1998,
	address = {北京},
	title = {晚清洋务运动事类汇钞五十七种},
	volume = {上册},
	publisher = {全国图书馆文献缩微复制中心},
	author = {佚名},
	year = {1998},
}

@book{赵景深1948,
	address = {上海},
	title = {文坛忆旧},
	publisher = {北新书局},
	author = {赵景深},
	year = {1948}
}

@book{邱陵1984,
	title = {书籍装帧艺术简史},
    address = {哈尔滨},
	publisher = {黑龙江人民出版社},
	editor = {邱陵},
    editortype={founder},
	year = {1984},
    pages={28-29}
}


@book{张树年1991,
	title = {张元济年谱},
    address = {北京},
	publisher = {商务印书馆},
	editor = {张树年},
    editortype={editor},
	year = {1991}
}


@newspaper{上海各路,
	address = {上海},
	title = {上海各路商界总联合会致外交部电},
	journal = {民国日报},
	date = {1925-08-14},
	number = {4},
}

@newspaper{四川会议厅暂行章程,
	chapter = {新章},
	title = {四川会议厅暂行章程},
	journal = {广益丛报},
    volume={第8卷第19期},
	date = {1910-09-03},
	note = {Issue: 第8卷第19期},
	pages = {1--2},
}

@archive{傅良佐致国务院电,
	title = {傅良佐致国务院电},
	date = {1917-09-15},
    number={北洋档案1011—5961,中国第二历史档案馆藏}
}

@book{谢兴尧1986,
	address = {西安},
	title = {荣庆日记},
	language = {zh},
	publisher = {西北大学出版社},
	year = {1986},
	editor= {谢兴尧},
    editortype={compiler}
}

@book{马克思恩格斯全集,
	address = {北京},
	title = {马克思恩格斯全集},
	volume = {31},
	publisher = {人民出版社},
	year = {1998},
}

@incollection{管志道1997,
	address = {济南},
	edition = {影印本},
	series = {四库全书存目丛书},
	title = {答屠仪部赤水丈书},
	volume = {2},
	booktitle = {续问辨牍},
	publisher = {齐鲁书社},
	author = {管志道},
	year = {1997},
    entrysubtype={classic},
	note = {Type: classic},
}

@article{何龄修1998,
	title = {读顾诚〈南明史〉},
	number = {3},
	journal = {中国史研究},
	author = {何龄修},
	year = {1998},
}

@incollection{黄仁宇1997,
	address = {北京},
	title = {为什么称为“中国大历史”?——中文版自序},
	booktitle = {中国大历史},
	publisher = {三联书店},
	author = {黄仁宇},
	year = {1997},
	pages = {2},
}

@book{蒋大兴2001,
	address = {北京},
	title = {公司法的展开与评判——方法·判例·制度},
	publisher = {法律出版社},
	author = {蒋大兴},
	year = {2001},
}

@book{金冲及1989,
	address = {北京},
	title = {周恩来传},
	publisher = {人民出版社、中央文献出版社},
	editor = {金冲及},
	year = {1989},
}

@newspaper{李眉1986,
	title = {李劼人轶事},
	journal = {四川工人日报},
	author = {李眉},
	date = {1986-08-22},
	number = {2},
}

@book{李鹏程1994,
	address = {北京},
	title = {当代文化哲学沉思},
	publisher = {人民出版社},
	author = {李鹏程},
	year = {1994},
    note={序言}
}

@incollection{唐振常1997,
	address = {上海},
	title = {师承与变法},
	booktitle = {识史集},
	publisher = {上海古籍出版社},
	author = {唐振常},
	year = {1997},
	pages = {65},
}

@online{王明亮1998,
	title = {关于中国学术期刊标准化数据库系统工程的进展},
	url = {http://www.cajcd.cn/pub/wml.txt/980810-2.html},
	urldate = {1998-10-04},
	author = {王明亮},
    date = {1998-08-16},
}

@article{魏丽英1990,
	title = {论近代西北人口波动的主要原因},
	number = {6},
	journal = {社会科学},
    address={兰州},
	author = {魏丽英},
	year = {1990},
	note = {Place: 兰州},
}


@article{费成康1999,
	title = {葡萄牙人如何进入澳门问题辨证},
	number = {9},
	journal = {社会科学},
    address={上海},
	author = {费成康},
	year = {1999},
	note = {Place: 上海},
}

@BOOK{李四1991b--,
  AUTHOR = {李四},
  PUBLISHER = {经济出版社},
  DATE = {1991},
  TITLE = {论计划与市场},
  edition={载于王五编辑《计划与市场》论文集},
  Pages  = {59-69},
  key={li3}
}


@book{Polo1997,
	address = {Hertfordshire},
	title = {The travels of {Marco} {Polo}},
	language = {en},
	publisher = {Cumberland House},
	author = {Polo, Marco},
	translator = {{William Marsden}},
	year = {1997},
    pages={58,88}
}



@incollection{楼适夷1988,
	address = {北京},
	edition = {增补本},
	title = {读家书,想傅雷(代序)},
	booktitle = {傅雷家书},
	publisher = {三联书店},
	author = {楼适夷},
	editor = {傅敏},
    editortype={reviser},
	year = {1988},
	pages = {2},
}

@incollection{鲁迅1981,
	address = {北京},
	title = {中国小说的历史的变迁},
	volume = {第9册},
	booktitle = {鲁迅全集},
	publisher = {人民文学出版社},
	author = {鲁迅},
	year = {1981},
	pages = {325},
}

@article{倪素香2002,
	title = {德育学科的比较研究与理论探索},
	number = {4},
	journal = {武汉大学学报},
    series={社会科学版},
	author = {倪素香},
	year = {2002},
	note = {Section: 社会科学版},
}

@article{麦孟华,
	title = {说奴隶},
	journal = {清议报},
	author = {伤心人(麦孟华)},
    volume={第69册,光绪二十六年十一月二十一日},
	note = {Volume: 第69册
Original Date: 光绪二十六年十一月二十一日},
	pages = {1},
}

@book{实藤惠秀1982,
	address = {香港},
	title = {中国人留学日本史},
	publisher = {香港中文大学出版社},
	author = {实藤惠秀},
	translator = {谭汝谦 and 林启彦},
	year = {1982},
}

@book{任继愈1983,
	address = {北京},
	title = {中国哲学发展史(先秦卷)},
	publisher = {人民出版社},
	editor = {任继愈},
    editortype={editor},
	year = {1983},
}

@book{狄葆贤,
	address = {上海},
	title = {平等阁笔记},
	publisher = {有正书局},
	author = {狄葆贤},
}

@phdthesis{方明东2000,
	address = {北京},
	type = {博士学位论文},
	title = {罗隆基政治思想研究(1913—1949)},
	school = {北京师范大学历史系},
	author = {方明东},
	year = {2000},
}

@incollection{佛克马1999,
	address = {北京},
	title = {走向新世界主义},
	booktitle = {全球化与后殖民批评},
	publisher = {中央编译出版社},
	author = {杜威·佛克马},
	editor = {王宁 and 薛晓源},
	year = {1999},
	pages = {247--266},
}

@book{Brooks2000,
	address = {Chicago},
	title = {Troubling confessions: {Speaking} guilt in law and literature},
	publisher = {University of Chicago Press},
	author = {Brooks, Peter},
	year = {2000},
}

@book{Starn1992,
	author = {Randolph Starn and Loren Partridge},
	title = {The Arts of Power: Three Halls of State in Italy, 1300-1600},
    address = {Berkeley},
	publisher = {California University Press},
	year = {1992},
    pages={19-28}
}


@archive{NixontoKissinger,
	title = {Nixon to {Kissinger}},
	date = {1969-02-01},
	number = {Box 1032, NSC Files, Nixon Presidential Material Project (NPMP), National Archives II, College Park, MD}
}


@incollection{Schfield1983,
	address = {Cambridge, Mass.},
	title = {The impact of scarcity and plenty on population change in {England}},
	booktitle = {Hunger and history: {The} impact of changing food production and consumption pattern on society},
	publisher = {Cambridge University Press},
	author = {Schfield, R. S.},
	editor = {Rotberg, R. I. and Rabb, T. K.},
    editortype={editor},
	year = {1983},
	pages = {55--88},
}

@book{清德宗实录,
    entrysubtype={classic},
	address = {北京},
	edition = {影印本},
	title = {清德宗实录},
	volume = {卷435,光绪二十四年十二月上},
	publisher = {中华书局},
	year = {1987},
	note = {Type: classic
Original Date: 光绪二十四年十二月上},
}

@book{广东通志,
    entrysubtype={classic},
    origyear={万历},
	address = {北京},
	edition = {影印本},
	title = {广东通志},
	volume = {卷15《郡县志二·广州府·城池》,《稀见中国地方志汇刊》},
	publisher = {中国书店},
	year = {1992},
}

@book{太平御览,
	address = {北京},
	edition = {影印本},
	title = {太平御览},
	volume = {卷690《服章部七》引《魏台访议》},
	publisher = {中华书局},
	year = {1985},
    entrysubtype={classic},
	note = {Type: classic
volume-title: 服章部七},
}

@book{旧唐书,
    entrysubtype={classic},
	address = {北京},
	edition = {标点本},
	title = {旧唐书},
	volume = {卷9《玄宗纪下》},
	publisher = {中华书局},
	year = {1975},
	note = {Type: classic
volume-title: 玄宗纪下},
}

@book{方苞集,
    entrysubtype={classic},
	address = {上海},
	edition = {标点本},
	title = {方苞集},
	volume = {卷6《答程夔州书》},
	publisher = {上海古籍出版社},
	year = {1983},
	note = {Type: classic
volume-title: 答程夔州书},
}

@archive{党外人士座谈会记录,
	title = {党外人士座谈会记录},
	date = {1950-07},
    number={李劼人档案,中共四川省委统战部档案室藏}
}


@newspaper{西南中委,
	address = {上海},
	title = {西南中委反对在宁召开五全会},
	journal = {民国日报},
	date = {1933-08-11},
	number = {第1张第4版},
}

@book{上海县续志,
    entrysubtype={classic},
    origyear={民国},
	title = {上海县续志},
	volume = {卷1《疆域》},
	note = {Type: classic
Original Date: 民国
volume-title: 疆域},
}

@book{嘉定县志,
    entrysubtype={classic},
    origyear={乾隆},
	title = {嘉定县志},
	volume = {卷12《风俗》},
	note = {Type: classic
Original Date: 乾隆
volume-title: 风俗},
}

@article{汪疑今1936,
	title = {江苏的小农及其副业},
	volume = {4},
	number = {6},
	journal = {中国经济},
	author = {汪疑今},
	date = {1936-06-15},
}

@inproceedings{任东来2000,
	address = {天津},
	title = {对国际体制和国际制度的理解和翻译},
	booktitle = {全球化与亚太区域化国际研讨会论文集},
	author = {任东来},
	date = {2000-06},
	pages = {9},
}

@book{毛祥麟1985,
	address = {上海},
	title = {墨余录},
	publisher = {上海古籍出版社},
	author = {毛祥麟},
	year = {1985},
    entrysubtype={classic},
	note = {Type: classic},
}

@article{黄义豪1997,
	title = {评黄龟年四劾秦桧},
	number = {3},
	journal = {福建论坛},
    series={文史哲版},
	author = {黄义豪},
	year = {1997},
	note = {Section: 文史哲版},
}


@article{章太炎演说,
 author    = {章太炎},
  title     = {在长沙晨光学校演说},
  date      = {1925-10},
}

@book{汤志钧长编,
author    = {汤志钧},
  title     = {章太炎年谱长编},
  volume={下册},
  date      = {1979},
  address   = {北京},
  publisher = {中华书局},
  pages={823}
}

@book{北京安徽会馆,
  title     = {北京安徽会馆志稿},
  date      = {2001},
  address   = {北京},
  publisher = {北京燕山出版社}
}



\end{filecontents}
    \addbibresource{\jobname.bib}
    %

\title{关于引文注释的规定}

%注意编者的类型:
%editor:主编
%compiler:整理
%redactor:校订
%reviser:修订
%founder:创建
%continuator:继承者
%collaborator:合作者

    \begin{document}

\maketitle

为便于学术交流和推进本社期刊编辑工作的规范化,在研究和 借鉴其他人文社会科学学术期刊注释规定的基础上,我们对原有引文注释规范进行了补充和完善,特制定新的规定。本规定适用于《中国社会科学》和《中国社会科学内部文稿》。

\section*{(一)注释体例及标注位置}

文献引证方式采用注释体例。
注释放置于当页下(脚注)。注释序号用①,②……标识,每页单独排序。正文中的注释序号统一置于包含引文的句子(有时候 也可能是词或词组)或段落标点符号之后。

\section*{(二)注释的标注格式}

\subsection*{1. 非连续出版物}

\subsubsection*{(1)著作}
标注顺序:责任者与责任方式/ 文献题名/ 出版地点/ 出版者/ 出版时间/ 页码。
责任方式为著时,“著”可省略,其他责任方式不可省略。
引用翻译著作时,将译者作为第二责任者置于文献题名之后。引用《马克思恩格斯全集》、《列宁全集》等经典著作应使用最新版本。
    
示例:
        
赵景深:《文坛忆旧》,上海:北新书局,1948 年,第 43 页。
\footfullcite[第 43 页]{赵景深1948}

谢兴尧整理:《荣庆日记》,西安:西北大学出版社,1986年,第 175 页。
\footfullcite[第 175 页]{谢兴尧1986}

蒋大兴:《公司法的展开与评判——方法·判例·制度》,北京:法律出版社,2001 年,第 3 页。
\footfullcite[第 3 页]{蒋大兴2001}

任继愈主编:《中国哲学发展史(先秦卷)》,北京:人民出版社,1983 年,第 25 页。
\footfullcite[第 25 页]{任继愈1983}

实藤惠秀:《中国人留学日本史》,谭汝谦、林启彦译,香港:中文大学出版社,1982年,第11—12页。
\footfullcite[第11—12页]{实藤惠秀1982}

金冲及主编:《周恩来传》,北京:人民出版社、中央文献出版社,1989年,第9页。
\footfullcite[第9页]{金冲及1989}

佚名:《晚清洋务运动事类汇钞五十七种》上册,北京:全国图书馆文献缩微复制中心,1998 年,第56页。
\footfullcite[第56页]{晚清洋务1998}

狄葆贤:《平等阁笔记》,上海:有正书局,出版时间不详,第 8 页。
\footfullcite[第 8 页]{狄葆贤}

《马克思恩格斯全集》第31卷,北京:人民出版社,1998年,第 46 页。
\footfullcite[第 46 页]{马克思恩格斯全集}


\subsubsection*{(2)析出文献}
标注顺序:责任者/ 析出文献题名/ 文集责任者与责任方式/ 文集题名/ 出版地点/ 出版者/ 出版时间/ 页码。
文集责任者与析出文献责任者相同时,可省去文集责任者。

示例:
杜威·佛克马:《走向新世界主义》,王宁、薛晓源编:《全球化与后殖民批评》,北京:中央编译出版社,1999年,第 247—266 页。
\footfullcite[第 247—266 页]{佛克马1999}

鲁迅:《中国小说的历史的变迁》,《鲁迅全集》第9册,北 京:人民文学出版社,1981 年,第325页。
\footfullcite[第325页]{鲁迅1981}

唐振常:《师承与变法》,《识史集》,上海:上海古籍出版社,1997年,第65页。
\footfullcite[第65页]{唐振常1997}

\subsubsection*{(3)著作、文集的序言、引论、前言、后记}

1)序言、前言作者与著作、文集责任者相同。

示例:

李鹏程:《当代文化哲学沉思》,北京:人民出版社,1994年,“序言”,第1页。
\footfullcite[“序言”,第1页]{李鹏程1994}

2)序言有单独的标题,可作为析出文献来标注。

示例:

楼适夷:《读家书,想傅雷(代序)》,傅敏编:《傅雷家书》(增补本),北京:三联书店,1988 年,第2页。
\footfullcite{楼适夷1988}

黄仁宇:《为什么称为“中国大历史”?——中文版自序》,《中国大历史》,北京:三联书店,1997年,第2页。
\footfullcite{黄仁宇1997}

3)责任者层次关系复杂时,可以通过叙述表明对序言的引证。为了表述紧凑和语气连贯,责任者与文献题名之间的冒号可省去,出版信息可括注起来。

示例:

见戴逸为北京市宣武区档案馆编、王灿炽纂《北京安徽会馆志稿》(北京:北京燕山出版社,2001年)所作的序,第2页。
\footnote{见戴逸为北京市宣武区档案馆编、王灿炽纂\citetitle{北京安徽会馆}(\citepub{北京安徽会馆})所作的序,第2页。}


\subsubsection*{(4)古籍}

1)刻本

标注顺序:责任者与责任方式/文献题名/卷次、篇名、部类(选项)/版本、页码。 部类名及篇名用书名号表示,其中不同层次可用中圆点隔开,原序号仍用汉字数字,下同。页码应注明 a、b 面。

示例:

姚际恒:《古今伪书考》卷3,光绪三年苏州文学山房活字本,第9页 a。
\footfullcite[第9页 a]{姚际恒}

2)点校本、整理本 标注顺序:

责任者与责任方式/文献题名/卷次、篇名、部类(选项)/出版地点/出版者/出版时间/页码。可在出版时间后 注明“标点本”、“整理本”等。

示例:

毛祥麟:《墨余录》,上海:上海古籍出版社,1985年,第35页。
\footfullcite[第35页]{毛祥麟1985}

3)影印本 标注顺序:

责任者与责任方式/文献题名/卷次、篇名、部类(选项)/出版地点/出版者/出版时间/(影印)页码。可在出版时间后注明“影印本”。为便于读者查找,缩印的古籍,引用页 码还可标明上、中、下栏(选项)。

示例:

杨钟羲:《雪桥诗话续集》卷5,沈阳:辽沈书社,1991年影印本,上册,第461 页下栏。
\footfullcite[上册,第461 页下栏]{杨钟羲1991}

《太平御览》卷690《服章部七》引《魏台访议》,北京:中华书局,1985年影印本,第3册,第3080页下栏。
\footfullcite[第3册,第3080页下栏]{太平御览}

4)析出文献

标注顺序:责任者/析出文献题名/文集责任者与责任方式/文集题名/卷次/丛书项(选项,丛书名用书名号)/版本或出版 信息/页码。

示例:

管志道:《答屠仪部赤水丈书》,《续问辨牍》卷2,《四库全书存目丛书》,济南:齐鲁书社,1997年影印本,子部,第88册,第73页。
\footfullcite[子部,第88册,第73页]{管志道1997}

5)地方志

唐宋时期的地方志多系私人著作,可标注作者;明清以后的地方志一般不标注作者,书名前冠以修纂成书时的年代(年号);民国地方志,在书名前冠加“民国”二字。新影印(缩印)的地方志 可采用新页码。

示例:

乾隆《嘉定县志》卷 12《风俗》,第7 页 b。
\footfullcite[第7 页 b]{嘉定县志}

民国《上海县续志》卷 1《疆域》,第10 页 b。
\footfullcite[第10 页 b]{上海县续志}

万历《广东通志》卷 15《郡县志二·广州府·城池》,《稀见中国地方志汇刊》,北京:中国书店,1992年影印本,第42册,第 367 页。
\footfullcite[第42册,第 367 页]{广东通志}


6)常用基本典籍,官修大型典籍以及书名中含有作者姓名的文集可不标注作者,如《论语》、二十四史、《资治通鉴》、《全唐文》、《册府元龟》、《清实录》、《四库全书总目提要》、《陶渊明集》等。

示例:

《旧唐书》卷9《玄宗纪下》,北京:中华书局,1975年标点本,第 233 页。
\footfullcite[第 233 页]{旧唐书}

《方苞集》卷6《答程夔州书》,上海:上海古籍出版社,1983年标点本,上册,第166 页。
\footfullcite[上册,第166 页]{方苞集}

7)编年体典籍,如需要,可注出文字所属之年月甲子(日)。

示例:

《清德宗实录》卷 435,光绪二十四年十二月上,北京:中华书局, 1987 年影印本,第 6 册,第 727 页。
\footfullcite[第 6 册,第 727 页]{清德宗实录}


\subsection*{2. 连续出版物}

\subsubsection*{(1)期刊}

标注顺序:责任者/ 文献题名/ 期刊名/ 年期(或卷期,出版年月)。
        刊名与其他期刊相同,也可括注出版地点,附于刊名后,以示区别;同一种期刊有两个以上的版别时,引用时须注明版别。

示例:

何龄修:《读顾诚〈南明史〉》,《中国史研究》1998 年第 3 期。
\footfullcite{何龄修1998}

汪疑今:《江苏的小农及其副业》,《中国经济》第 4 卷第 6 期, 1936 年 6 月 15 日。
\footfullcite{汪疑今1936}

魏丽英:《论近代西北人口波动的主要原因》,《社会科学》(兰州)1990 年第 6 期。
\footfullcite{魏丽英1990}

费成康:《葡萄牙人如何进入澳门问题辨证》,《社会科学》(上海)1999 年第 9 期。
\footfullcite{费成康1999}

董一沙:《回忆父亲董希文》,《传记文学》(北京)2001年第3 期。

李济:《创办史语所与支持安阳考古工作的贡献》,《传记文学》(台北)第 28 卷第 1 期,1976 年 1 月。

黄义豪:《评黄龟年四劾秦桧》,《福建论坛》(文史哲版)1997年第3期。
\footfullcite{黄义豪1997}

苏振芳:《新加坡推行儒家伦理道德教育的社会学思考》,《福建论坛》(经济社会版)1996 年第 3 期。

叶明勇:《英国议会圈地及其影响》,《武汉大学学报》(人文科学版)2001 年第 2 期。

倪素香:《德育学科的比较研究与理论探索》,《武汉大学学报》(社会科学版)2002 年第 4 期。
\footfullcite{倪素香2002}

\subsubsection*{(2)报纸}

标注顺序:责任者/ 篇名/ 报纸名称/ 出版年月日/ 版次。
早期中文报纸无版次,可标识卷册、时间或栏目及页码(选注项)。同名报纸应标示出版地点以示区别。

示例:

李眉:《李劼人轶事》,《四川工人日报》1986 年 8 月 22 日,第 2 版。
\footfullcite{李眉1986}

伤心人(麦孟华):《说奴隶》,《清议报》第 69 册,光绪二十六年十一月二十一日,第 1 页。
\footfullcite{麦孟华}

《四川会议厅暂行章程》,《广益丛报》第 8 年第 19 期,1910 年9 月 3 日,“新章”,第 1—2 页。
\footfullcite[“新章”,第 1—2 页]{四川会议厅暂行章程}

《上海各路商界总联合会致外交部电》,《民国日报》(上海)1925 年 8 月 14 日,第 4 版。
\footfullcite{上海各路}

《西南中委反对在宁召开五全会》,《民国日报》(广州)1933年8月 11 日,第1 张第4 版。
\footfullcite{西南中委}

\subsection*{ 3.  未刊文献}

\subsubsection*{(1)学位论文、会议论文等}

标注顺序:责任者 / 文献标题 / 论文性质 / 地点或学校 / 文献形成时间 / 页码。

示例:

方明东:《罗隆基政治思想研究(1913—1949)》,博士学位论文,北京师范大学历史系,2000 年,第 67 页。
\footfullcite{方明东2000}

任东来:《对国际体制和国际制度的理解和翻译》,全球化与亚太区域化国际研讨会论文,天津,2000 年 6 月,第 9 页。
\footfullcite{任东来2000}

\subsubsection*{(2)手稿、档案文献}

标注顺序:文献标题 / 文献形成时间 / 卷宗号或其他编号 / 收藏机构或单位。

示例:

《傅良佐致国务院电》,1917 年 9 月 15 日,北洋档案1011—5961,中国第二历史档案馆藏。
\footfullcite{傅良佐致国务院电}

《党外人士座谈会记录》,1950 年7月,李劼人档案,中共四川省委统战部档案室藏。
\footfullcite{党外人士座谈会记录}


\subsection*{ 4. 转引文献}

无法直接引用的文献,转引自他人著作时,须标明。标注顺序:责任者/原文献题名/原文献版本信息/原页码(或卷期)/转引文献责任者/转引文献题名/版本信息/页码。

示例:

章太炎:《在长沙晨光学校演说》,1925 年 10 月,转引自汤志钧:《章太炎年谱长编》下册,北京:中华书局,1979 年,第 823 页。
\footnote{\fullinnercite{章太炎演说},转引自\fullcite{汤志钧长编}}

另一种方式:\footfullcite{李四1991b--}


\subsection*{5. 电子文献}

电子文献包括以数码方式记录的所有文献(含以胶片、磁带等介质记录的电影、录像、录音等音像文献)。
标注项目与顺序:责任者/电子文献题名/更新或修改日期/ 获取和访问路径/引用日期。
示例:

王明亮:《关于中国学术期刊标准化数据库系统工程的进展》,1998年8月16日, \url{http://www.cajcd.cn/pub/wml.txt/980810-2.html},1998 年 10 月4日。 
\footfullcite{王明亮1998}

扬之水:《两宋茶诗与茶事》,《文学遗产通讯》(网络版试刊)2006年 第1期, \url{http://www.literature.org.cn/Article.asp?ID=199},2007 年 9 月 13 日。
\footfullcite{扬之水2007}

\subsection*{6.  外文文献}
(1)引证外文文献,原则上使用该语种通行的引证标注方式。

(2)本规范仅列举英文文献的标注方式如下:

1)专著

标注顺序:责任者与责任方式/ 文献题名/ 出版地点/ 出版者/ 出版时间/ 页码。文献题名用斜体,出版地点后用英文冒号,其余各标注项目之间,用英文逗点隔开,下同。

示例:

Peter Brooks, \emph{Troubling Confessions: Speaking Guilt in Law and Literature}, Chicago: University of Chicago Press, 2000, p. 48.
\footfullcite{Brooks2000}


Randolph Starn and Loren Partridge, \emph{The Arts of Power: Three Halls of State in Italy, 1300-1600}, Berkeley: California University Press, 1992, pp.19-28.
\footfullcite{Starn1992}

2)译著

标注顺序:责任者/ 文献题名/ 译者/ 出版地点/ 出版者/ 出版时间 /页码。

示例:

M. Polo, \emph{The Travels of Marco Polo}, trans. William Marsden, Hertfordshire: Cumberland House, 1997, pp. 55, 88.
\footfullcite{Polo1997}

 3)期刊析出文献

标注顺序:责任者/ 析出文献题名 /期刊名/ 卷册及出版时间/ 页码。析出文献题名用英文引号标识,期刊名用斜体,下同。

示例:

Heath B. Chamberlain, “On the Search for Civil Society in China”,
\emph{Modern China}, vol. 19, no. 2 (April 1993), pp. 199-215.
\footfullcite{Chamberlain1993}

4)文集析出文献

标注顺序:责任者/析出文献题名/ 文集题名/ 编者/ 出版地点/出版者/ 出版时间/ 页码。

示例:

R. S. Schfield, “The Impact of Scarcity and Plenty on Population Change in England,” in R. I. Rotberg and T. K. Rabb, eds., \emph{Hunger and History: The Impact of Changing Food Production and Consumption Pattern on Society}, Cambridge, Mass.: Cambridge University Press, 1983, p. 79.
\footfullcite{Schfield1983}

5)档案文献

标注顺序:文献标题/ 文献形成时间/ 卷宗号或其他编号/ 藏所。

Nixon to Kissinger, February 1, 1969, Box 1032, NSC Files, Nixon Presidential Material Project (NPMP), National Archives II, College Park, MD.
\footfullcite{NixontoKissinger}



\section*{(三)其他}

\subsection*{1. 再次引证时的项目简化}

同一文献再次引证时只需标注责任者、题名、页码,出版信息可以省略。

示例:

赵景深:《文坛忆旧》,第 24 页。
\footfullcite[第 24 页]{赵景深1948}

鲁迅:《中国小说的历史的变迁》,《鲁迅全集》第 9 册,第326 页。
\footfullcite[第326 页]{鲁迅1981}

\subsection*{2. 间接引文的标注}
间接引文通常以“参见”或“详见”等引领词引导,反映出与正文行文的呼应,标注时应注出具体参考引证的起止页码或章节。标注项目、顺序与格式同直接引文。

示例:
参见邱陵编著:《书籍装帧艺术简史》,哈尔滨:黑龙江人民出版社,1984 年,第 28—29 页。
\footnote{参见\fullcite{邱陵1984}}

详见张树年主编:《张元济年谱》,北京:商务印书馆,1991年,第 6 章。
\footnote{详见\fullcite[第 6 章]{张树年1991}}


\subsection*{3. 引用先秦诸子等常用经典古籍}

可使用夹注,夹注应使用不同于正文的字体。

 示例 1:

 庄子说惠子非常博学,“惠施多方,其书五车。”\citejz{庄子天下}

 示例 2:

 天神所具有道德,也就是“保民”、“裕民”的道德;天神所具有的道德意志,代表的是人民的意志。这也就是所谓“天聪明自我民聪明,天明畏自我民明畏”\citejz{尚书皋陶谟},“民之所 欲,天必从之”\citejz{尚书泰誓}。









\newpage
{
%\hyphenation{Proce-edings}
\hyphenpenalty=100 %断词阈值, 值越大越不容易出现断词
\tolerance=5000 %丑度, 10000为最大无溢出盒子, 参考the texbook 第6章
\hbadness=100 %如果丑度超过hbadness这一阀值, 那么就会发出警告
    \printbibliography

}

    \end{document} 